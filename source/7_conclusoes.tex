\chapter{Considerações Finais} \label{sec:consideracoesFinais}

Primeiramente, é relevante ressaltar as limitações encontradas ao longo do trabalho.
Cita-se que os arquivos georreferenciados dos bairros de ambos os estudos de caso continham apenas municípios de maior porte, os quais costumam ter bibliotecas digitais mais acessíveis.
As cidades satélite de menor porte, na maioria das vezes, não apresentavam arquivos consolidados de mapeamento dos bairros.
Assim, o trabalho enfrentou em ambos os estudos uma redução das áreas de estudo para qualquer análise de organização espacial a partir de bairros.
Uma estratégia para mitigar este efeito seria a adoção de técnicas como as apresentadas por \citeonline{merchan2020quantifying} quando propõe o agrupamento (clusterização) de regiões como forma alternativa de análise.

Ademais, destaca-se que ambos os estudos de caso trabalharam com redes de entregas logísticas associadas a espaços de alta urbanização e pertencentes a empresas de grande porte.
Entende-se que em condições diferentes, como zonas rurais, novas avaliações seriam necessárias para a validação da metodologia desenvolvida.
Também destaca-se que a estrutura das malhas viárias pode ser alterada com o passar dos anos (\citeauthoronline{}, \citeyear{}), o que abre oportunidade para se revisitar este estudo num horizonte de 10, 20 ou 30 anos. 

Ainda nas limitações, entende-se que muitos fatores afetam a performance das rotas de entrega e, dentre eles, há fatores que não estavam contemplados nas bases de dados ou nos dados abertos utilizados no trabalho.
Elementos como a disponibilidade de vagas para estacionamento, o fluxo nas vias e o tipo de fachada/portaria são limitantes à qualidade dos resultados obtidos.
Tais considerações podem, porém, servir como objeto de estudo para trabalhos futuros.

Já no espectro dos resultados obtidos, estabelece-se que os problemas associados às entregas de última milha são complexos e podem estar associados a múltiplos fatores ligados à equipe de entregas, ao cliente, à empresa e suas políticas de estímulo, à entrega em si, à localização e ao meio onde a rota se dá e o cliente se localiza.
Dessa forma, não é possível estabelecer uma única variável que seja capaz de predizer a ocorrência não aderência ao sequenciamento de entregas ou que possa guiar o processo de tomada de decisões das empresas encarregadas dessa distribuição.
Por exemplo, observa-se, a partir do estudo de caso da empresa de bebidas brasileira que vários dos fatores correlatos aos problemas de entregas, tais como horário da entrega, volume da entrega e equipe responsável são fatores influentes que podem ser diretamente influenciados pela própria empresa.

Já quanto aos parâmetros relativos à malha viária em que as entregas de última milha estudadas estavam inseridas, o que se viu foi a confirmação de algumas das hipóteses levantadas previamente com base na literatura. 
De fato, todas as análises da empresa de bebidas brasileira apontaram para uma heterogeneidade das características de ``circuicidade'', conectividade e orientação ao longo dos bairros estudados.
Nesse sentido, resultados sugeriram que há tendência de haver maior quantidade de devoluções ou repasses conforme se aumenta a complexidade da malha.

Através do estudo de caso da \textit{Amazon}, identifica-se que o local de entregas e a estrutura urbana apresentam efeitos sobre os problemas de entrega antes pouco trabalhados na perspectiva acadêmica e, de acordo com a visita técnica à empresa de bebidas, na perspectiva corporativa.
%
Tais efeitos, menos intuitivamente puderam ser trabalhados com maior detalhe no estudo de caso da \textit{Amazon} devido à qualidade e ao volume das informações disponibilizadas. No estudo, identificou-se uma série de variáveis relativas à malha viária, à estrutura urbana e à configuração das rotas que poderiam associar-se como efeitos causadores às variáveis-problema das entregas. Após uma série de análises de correlação e correlação múltipla, foi possível estabelecer que 3 fatores afetam em 18\% combinadamente a ocorrência de repasses. Tais fatores, direta ou indiretamente relacionados à concepção da estrutura viária e da densidade de entregas na região têm potenciais benefícios à empresa, à sociedade e ao cliente que excedem à simples ocorrência de repasses na rota de entregas.

Como mencionado anteriormente, a não aderência ao sequenciamento programado de entregas acarreta num tempo de circulação maior, com uma distância maior de circulação, relacionados a maiores custos e impactos ambientais não previstos no planejamento da empresa. 
Sendo assim, os efeitos das três variáveis utilizadas para representar ocorrência de repasses podem alterar esse cenário em até 18\% conjuntamente.
Tal resultado é ainda mais relevante ao se considerar que operações de distribuição de última milha compreendem setores de grande porte ao redor do mundo.
%
Estes efeitos, por configurarem em grande parte efeitos de estrutura urbana, podem ser considerados por entidades públicas (e.g. prefeituras) como oportunidades de investimento visando o futuro e o crescimento das empresas de logística locais, aumentando a qualidade da entrega para o operador e para o cliente.
Por outro lado, tais efeitos podem ser considerados pelas próprias empresas logísticas em seu planejamento estratégico, a fim de se tomar decisões acerca de quais regiões operar seu sistema de entregas considerando-se tais variáveis.

Assim, no que tange às hipóteses preliminares estabelecidas anteriormente, identifica-se que suas validades foram parcial ou integralmente confirmadas.
O primeiro resultado listado teve sua confirmação parcial, dado que a configuração das vias comprovou-se como um elemento que pode fazer das cidades lugares mais complexos para entregas, apesar de que pouco se trabalhou sobre os efeitos das condições de terreno, como citava-se.
Já o segundo resultado teve sua validação integral, dado que foi possível quantificar as condições das malhas urbanas para realizar comparações baseadas em estatísticas e  investigações de correlação. 
O terceiro resultado, por sua vez, teve uma comprovação parcial, pois os resultados de correlação de NAS com fator de circuito foram interessantes, porém insuficientes como explicação única do fenômeno.
E, por fim, o quarto resultado esperado teve, também, uma validação parcial, dado que foi possível estabelecer uma relação entre o NAS e os indicadores de malha viária, porém, assim como o fator de circuito, foram insuficientes quando analisados isoladamente.

A documentação dos métodos aplicados também se mostrou satisfatória, tendo inclusive seu acesso disponibilizado abertamente online, o que facilita a transparência e confiabilidade dos resultados.
Os códigos utilizados foram construídos de forma que futuras análises poderão ser realizadas sem necessidade de grandes alterações, bastando apenas carregar os dados relativos a uma nova região de estudo.
Nesse sentido, identifica-se que os métodos criados tem a potencialidade de tornarem-se ferramentas (ou serem incorporados a ferramentas já existentes) para aplicações futuras.

Ao final do trabalho, sugere-se, em caso de trabalhos futuros revisitarem o mesmo tema, adoção de modelos não supervisionado de aprendizado de máquina (\textit{machine learning}), assim como feito por, como forma de encontrar correlações não previstas anteriormente.
Finalmente, é importante destacar que os resultados do trabalho podem ser estendidos para além da logística, como educação, saúde, urbanismo, gestão pública, mobilidade urbana, segurança pública, 
% Esse espectro de áreas beneficiadas é devido ao impacto que a logísitca de última, e seu planejamento adequado, tem sobre diversas áreas.