\chapter{Caracterização das empresas analisadas} \label{sec:Contexto}

Conforme mencionado na Seção \ref{objetivos}, o presente trabalho será constituído de dois estudos de caso com o objetivo de se aplicar a metodologia desenvolvida em situações operacionais de entrega reais.
O primeiro estudo de caso é baseado numa empresa de entregas de bebidas brasileira cuja identificação foi tornada anônima e o segundo é baseado na empresa estadunidense \textit{Amazon.com}. 
Desta forma, o objetivo do presente capítulo é apresentar as características de ambas as empresas de modo a contextualizar as análises, hipóteses e resultados obtidos nos capítulos posteriores.

%%%%%%%%%%%%%%%%%%%%%%%%%%%%%%%%%%%%%%%%%%%%%%%%%%%%%%%%%%%%%%%%%%%%
\section{Empresa de bebidas brasileira} \label{sec:realidade_empresa}

Diferentemente do segundo estudo de caso, sua identificação teve de ser tornada anônima devido às especificidades do caso.
O estudo foi estabelecido através de uma base de dados histórica fornecida e uma visita técnica realizada a um de seus CDs, que serão apresentados a posteriori.

A empresa é caracterizada por se destacar internacionalmente na fabricação e distribuição de bebidas, com grande relevância na indústria alimentícia nacional.
Sua operação é caracterizada por uma diversidade de instalações (fábricas, CDs, armazéns, etc.) e um portfólio de diversos produtos ou \textit{Stock-Keeping-Units} (SKUs) diferentes, contando com mais de 30 mil funcionários no Brasil e uma produção de mais de 15 bilhões de litros de bebidas por ano.
Os clientes da empresa são majoritariamente bares, restaurantes e supermercados. Sendo assim, a distribuição de última milha da empresa pode ser caracterizada como \textit{Business-to-business} (B2B), visto que os clientes não são os consumidores finais dos produtos.

A empresa atual nos diferentes processos da indústria de bebidas, desde a sua fabricação até a entrega final nos bares, restaurantes e locais de venda em geral.
Assim, sua operação logística é caracterizada por um nível de complexidade elevado associado a várias etapas  de deslocamento, desde a saída das fábricas de bebidas até o cliente.
Tais etapas contam com diferentes níveis de planejamento e intensidade operacional, marcados por volumes grandes e percursos longos nas etapas de transferências entre instalações físicas, por exemplo, e volumes pequenos, em veículos urbanos com distância reduzidas, na etapa final de entregas, por exemplo.
A etapa final de entregas é o trecho entre o CD final de armazenamento (normalmente localizados perifericamente às grandes aglomerações urbanas) e os clientes finais.
Assim, cita-se a operação como bastante volátil a depender dos clientes e de suas oscilações de consumo e mais pulverizada, considerando-se a alta quantidade de pontos de entregas finais que devem ser, na maioria das vezes, atendidos com veículos menores devido às regras de circulação dos centros urbanos.

Tal etapa, em maior detalhamento, é realizada a partir de diversos CDs no Brasil, podendo, em certas circunstâncias, ter mais de um CD para a mesma região urbana.
Esse é o caso da Região Metropolitana de São Paulo.
Nela, a empresa situa-se através de um conjunto de CDs localizados nas regiões mais periféricas, devido ao custo fixo das instalações ser mais reduzido.
Cada CD é responsável por uma parcela geográfica da Região Metropolitana, de modo a satisfazer toda a demanda pulverizada dos municípios que a compõe.
O presente trabalho estuda dois CDs presentes na Região Metropolitana citada, a fim de se aplicar a metodologia desenvolvida nessa última etapa de entregas da empresa.
O primeiro CD é estudado através de uma visitação técnica realizada pelos Autores no início da pesquisa.

Ao longo da visita pode-se contemplar o passo a passo pelo qual se dá a rota de entregas de um dos veículos padrão do CD em associação com a observação dos comportamentos e tomadas de decisões dos clientes, da equipe de entregas e da equipe de supervisão.
Nesse aspecto, cita-se que a empresa conta com uma estrutura hierárquica bastante marcada e, no caso da entrega existem três cargos relevantes a serem citados. 

O primeiro cargo é o de supervisor de rotas. Seu trabalho é manter o controle e o contato direto com todas as equipes de entrega pelas quais ele é responsável.
Eles dispõe de veículos próprios para acompanhamento presencial das equipes e utilizam o telefone como plataforma básica de comunicação e acompanhamento. Em geral, no CD em questão, os supervisores de rota são responsáveis por 20 a 30 equipes de entrega cada um.
Em sequência, há os dois cargos das equipes.

O segundo cargo é o de motoristas. 
O motorista é tido como o líder da equipe responsável pela direção do veículo, controle da rota de entregas e principal tomador de decisões.
Normalmente, o motorista é o mais experiente da equipe e é o que mantém contato direto com o supervisor e com o cliente.

Por fim, a equipe é complementada com os ajudantes.
Cada equipe conta com pelo menos um ajudante, sendo que pode haver um segundo caso a rota de entregas do dia tenha mais de 230 caixas a serem entregues.
O ajudante costuma ser o mais inexperiente da equipe, cujas responsabilidades são associadas ao trabalho mais manual da entrega, realizando a abertura do caminhão, a separação dos produtos do cliente em questão, a entrega em si e o fechamento do caminhão.
Sua relação direta é estabelecida mais com o motorista da equipe, mas o responsável formal pelos ajudantes é o supervisor.

Assim, a visitação técnica realizada visou conhecer mais detalhadamente o planejamento tático de roteirização das entregas, bem como as características e nuances presentes no cotidiano da operação.
Tratando-se de um mercado de alta demanda, toda a rotina operacional de entregas apresenta ciclicidade diária, geralmente funcionando de segunda a sábado e integrando diversas áreas da empresa que contribuem com distribuição de última milha.

O segundo CD estudado é compreendido através de uma base de dados histórica das rotas de entregas de todos os seus veículos num período de 5 meses (entre janeiro e maio de 2015).
Sua operação de entregas é numa área que engloba três cidades da Região Metropolitana de São Paulo.
O CD conta com uma operação de larga escala  baseada em mais de 40 veículos e mais de 4.500 clientes atendidos.
Assim, ele é parte essencial da composição final de entregas das bebidas na maior região metropolitana do Brasil. 

Identifica-se que a base contempla 4.554 clientes diferentes, que serão aqui identificados como Ponto de Entrega (PDE).
Estes clientes são identificados cada um com um número aleatório diferente, de modo a se preservar o anonimato de cada PDE.
%
Ademais, a base apresenta um conjunto de 42 veículos de entrega, cada um identificado por um número aleatório que representa o número da placa do veículo.

No total observa-se uma média de 694 PDEs atendidos por dia, a partir de uma média de 38,85 veículos utilizados por dia, durante um período total de 124 dias.
Assim, a base concretiza-se com um total de 88.338 entregas, as quais estão divididas em rotas.
Uma rota é única para cada dia, veículo e conjunto de clientes, portanto se um mesmo veículo atender exatamente os mesmos clientes em dois dias seguidos, serão consideradas duas rotas distintas.

A base é discretizada na chave de entregas, apresentando informações relativa a cada entrega realizada ao longo do período descrito.
Cada entrega é relacionada com 42 variáveis identificatórias e descritivas que singularizam cada entrega.
A principal variável presente na base diz respeito à ordem programada e ordem real de cada entrega na rota, assim permitindo a comparação entre rotas programadas e rotas realmente realizadas.

Uma limitação que pode ser observada na base é a falta de informações relativas ao PDE e à equipe (motorista e ajudante(s)) de entrega.
No total, no que tange o PDE, dispõe-se apenas do código identificador unitário e das suas coordenadas geodésicas decimais (latitude e longitude), limitando a compreensão mais aprofundada das características relativas a cada ponto de entrega, embora preserve o anonimato dos clientes.
O volume e a frequência de entregas, porém, são informados, o que permitirá atribuir noções indiretas de tamanho e relevância de cada PDE. 
No que tange a equipe de entregas, sabe-se, apenas, a placa do veículo que realizou cada rota.
Ademais, a informação de capacidade de cada veículo não foi divulgada.

Uma dificuldade adicional enfrentada foi a falta de precisão aplicada às informações geográficas.
A base conta com três dados de localização para cada PDE que influenciam diferentemente cada resultado.
Esses valores variam dado que um mesmo PDE pode ser atendido em mais de um dia e, assim, ter coordenadas de entregas cadastradas levemente diferentes.
Em outras palavras, a cada dia em que se realiza uma entrega, o veículo pode estacionar em uma posição distinta da rua, e, assim, registrar como coordenadas de entrega um valor diferente. 
Todavia, uma média das coordenadas registradas na base foi utilizada de modo a facilitar o estudo. 

Adicionalmente, a Tabela \ref{tab:Variaveis_BD}, disponível na seção apêndice \ref{sec:AppBDAmbev}, ilustra as variáveis apresentadas na base, bem como o intervalo de valores de cada uma das variáveis.

%%%%%%%%%%%%%%%%%%%%%%%%%%%%%%%%%%%%%%%%%%%%%%%%%%%%%
\section{Empresa americana \textit{Amazon.com, Inc.}}

Enquanto se aproximavam da segunda metade do trabalho, os autores se depararam com a oportunidade de entrar em contato com uma base de dados históricos de entregas de última milha nos Estados Unidos de uma das maiores empresas de comércio eletrônico no mundo, a \textit{Amazon.com, Inc.}.
A empresa apresenta uma das operações de entregas mais complexas do planeta, contando com fornecedores primários e clientes finais presentes em mais de 19 países em todos os continentes. Assim, como cita \citeonline{Holden2020}, a empresa é altamente dependente de otimização de rotas, a fim de se maximizar produtividade da empresa.

Assim como na empresa anterior, sua operação é estabelecida em uma sequência de etapas desde a saída do produto do local de despacho até a recepção do cliente final.
Assim, a empresa conta com uma gama de políticas específicas à operação de cada região e de cada classe de produtos.
Em exemplo, cita-se os produtos de alta demanda que, em muitos casos, já ficam à disposição nos CDs finais das regiões de grande consumo para garantir entregas de curto prazo ao custo de um armazenamento maior e uma estrutura logística \textit{inbound} do CD com maior complexidade.
Após a saída do CD, os produtos a serem entregues diariamente nas regiões altamente urbanizadas estão sujeitos a rotas de maior densidade.
Os caminhões da Amazon transportam mais de 2.000 caixas por rota, como cita \citeonline{Taylor2019}.

Preocupada em otimizar seu desempenho operacional, a empresa publicou em 2021, sob parceria com o \textit{Massachusetts Institute of Technology} (MIT), um desafio global de roteirização.
O desafio em questão visava estimular a contribuição por parte do meio acadêmico para a operação da empresa, permitindo que diversos modelos e propostas de melhorias pudessem ser testados.
Algumas dessas propostas foram consolidadas e então publicadas por \citeonline{winkenbach2021technical}. 

Adicionalmente, destaca-se que a dinâmica de entregas da empresa norte-americana se diferencia da empresa brasileira. 
Primeiro, cita-se que a Amazon contempla distribuição de entregas de modelo \textit{Business-to-consumer} (B2C), em que os receptores das entregas são os consumidores finais, enquanto o primeiro estudo de caso se encaixava no modelo B2B.
Além disso, os casos também diferenciam-se devido ao volume e dimensões dos pacotes entregues, bem como pela segurança da entrega.
Nos Estados Unidos, o cliente pode, em geral, contar com a segurança da entrega sem a necessidade de estar presente para assinar a recepção do bem, algo que no Brasil não se aplica.
Além disso, cita-se que os casos diferenciam-se, também, pelos meios de pagamento.
As entregas de bebidas analisadas no primeiro estudo se caracterizavam pelo fato de o pagamento da nota fiscal ser realizado instantes antes da entrega ser efetivada, geralmente logo depois que o veículo se apresenta à portaria do cliente.
Com o a empresa estadunidense, contudo, os pagamentos já estão confirmados desde instantes após a compra, não correndo o risco de alguma entrega ser devolvida por indisponibilidade do cliente em realizar o pagamento.

Ademais, como citado, foi obtida uma base de dados histórica da operação de entregas da empresa através do desafio global de roteirização que publicou em 2021.
Estas entregas contemplam diferentes produtos, de diferentes dimensões, não somente bebidas.
A base está publicada pelo programa de dados abertos - \textit{Open-data} - da empresa e a descrição detalhada da estruturação dos dados é apresentada por \citeonline{amz_lmrc_Merchan2022}.
Destaca-se que, por fazer parte de um programa de dados abertos, a organização destes conjuntos de dados, bem como a precisão das informações, superam com facilidade a base de dados estudada anteriormente.

A única limitação, porém, é que neste caso não estão disponíveis os dados de rotas programadas, ou seja, conta-se apenas com um histórico de rotas de entregas realizadas para determinadas regiões durante um determinado período de tempo.
Uma descrição completa dos dados fornecidos pode ser encontrada na seção de apêndice \ref{sec:appDBAmazon}.