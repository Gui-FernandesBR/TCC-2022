\begin{resumo}

A distribuição de entregas de última milha em zonas urbanas tem se destacado como uma das atividades mais importantes da logística nos últimos anos.
Caracterizado por abranger a última etapa do processo de entregas de larga escala, o segmento de última milha apresenta uma grande dinamicidade diária, com alto impacto no desempenho das empresas.
Assim, para se aperfeiçoar ou otimizar tal setor, é relevante que se possa ter um bom planejamento tático de suas operações de entrega.
Um problema que se observa, porém, é a diferença entre rotas realizadas e rotas programadas, chamada ``não aderência ao sequenciamento de entregas''.
Essa diferença pode ser avaliada a partir de três indicadores: Repasse, Não Aderência Sequencial e Devolução.
Estes fenômenos compreendem comportamentos inesperados no roteiro de entregas, que implicam em uma série de impactos econômicos, sociais e ambientais.
%
Para que se tenha maior controle sobre a não aderência, é relevante que se tenha conhecimento de quais fatores influenciam sua ocorrência.
O presente trabalho apresenta os primeiros passos para a identificação da correlação entre tais variáveis e diferentes fatores de potencial causalidade.
%
Foram realizados dois estudos de caso com dados reais de entregas de última milha a partir dos Centros de Distribuição respectivos de cada empresa.
Um dos estudos é relativo a uma empresa de bebidas brasileira de grande porte com histórico de entregas na Região Metropolitana de São Paulo, enquanto o outro é do histórico de entregas da \textit{Amazon.com} nos Estados Unidos.
%
Dessa forma, o trabalho permitiu análises de correlação que englobam medidas de caracterização das entregas, de autocorrelação espacial, do posicionamento relativo dos pontos de entrega, do fator de circuito e de um conjunto de variáveis capazes de mensurar diferentes aspectos da geometria de malhas viárias.
% 
A partir dos resultados foi possível identificar efeitos correlatos de aspectos das entregas, tais como o horário da entrega e a equipe que o realiza, assim como com as características associadas à malha viária.
Por meio de regressões lineares simples e múltiplas, foi possível estabelecer que três indicadores associados ao tamanho da rota, ao fator de circuito e à orientação das vias afetam em 18\%, combinadamente, a ocorrência de repasses.
As contribuições do trabalho podem ser úteis para diferentes áreas de estudo, incluindo educação, saúde e gerenciamento de tráfego.
O trabalho se baseou majoritariamente em ferramentas de código aberto disponibilizadas tanto por trabalhos anteriores quanto pelos próprios autores.
Por exemplo, o pacote \textit{lmr\_analyzer} foi escrito em linguagem \textit{Python} e sob demanda para este trabalho, e ao final foi disponibilizado nas principais plataformas de código aberto atuais, permitindo a aplicação dos métodos em outros contextos e também a reprodução dos resultados obtidos.  
\\[3\baselineskip]
%
\textbf{Palavras-Chave} -- Logistics, Last-mile distribution, Vehicle routing problem, Street Network Analysis, Circuity.
\end{resumo}


% ========== Abstract ==========
\begin{abstract}

% In recent years, a sector that has stood out in the logistics area is the \textit{last-mile} segment.
The last mile distribution plays a key role in supply chain management nowadays.
Characterized by covering the last stage of the large-scale delivery process, the last mile segment presents large dynamics on a daily basis, with a high impact on the overall performance of deliveries.
Therefore, in order to either improve or optimize last mile deliveries, it is crucial to have a good tactical planning of your operation.
This paper has ellaborated on top of the phenomena known as ``non-adherence between planned and executed routes'' on last mile distribution.
%
This non-adherence can be evaluated based on three key performance indicators: Failed delivery attempts, Sequential Non-Adherence and Rejections.
These phenomena represent unexpected behaviors in the delivery route and imply in several economic and environmental impacts.
%
With that in mind, to provide a greater control of last mile operations, it is important understand which indicators may influence the occurrence of the non-adherence.
To accomplish with that, this paper presents the first steps towards identifying the correlation between the indicators and different potential causality factors.
%
Therefore, two different case studies with actual data on last mile deliveries were performed. 
One of the studies is related to a large Brazilian beverage company with a database of deliveries in the São Paulo Metropolitan Region (Brazil), while the other is about the delivery database of \textit{Amazon.com} in the United States.
%
Our contributions has allowed for an extensive set of correlation analyzes that address measures of characterizing deliveries routes, spatial autocorrelation, the relative positioning of dropoff points, the circuity factor and a set of variables for measuring different aspects of the street network topology.
%
Furthermore, the results suggests the existence of correlated effects between elements related to deliveries, such as the time of delivery and the delivery team that performs it, as well as characteristics associated with the street network.
After series of both simple and multi-variable regressions, it was possible to establish that three indicators associated with the size of the route, the circuity factor and the orientation may directly affect 18\% of the occurrence of failed delivery attempts.
These contributions may be useful to different study areas, including education, health, traffic management. 
The major part of this work has been based on top of open-source tools provided by previous works and also by the authors.
For instance, the Python module \textit{lmr\_analyzer}, created during this work, is now totally available on open-source platforms, allowing for further investigations under different contexts, as well as the reproduction of this work's results.  
\\[3\baselineskip]
%
\textbf{Keywords} -- Logistics, Last-mile distribution, Vehicle routing problem, Street Network Analysis, Circuity.
\end{abstract}